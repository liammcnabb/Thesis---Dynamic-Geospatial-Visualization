\usepackage[table,dvipsnames]{xcolor}
\usepackage{hyperref}
\hypersetup{
	colorlinks = true,
	allcolors = [rgb]{0.0902,0.0784,1},	
	linkcolor = [rgb]{0.1647,0.0980,0.6583},	
}

\usepackage{bibentry}
\nobibliography*



\usepackage{multicol,lipsum}
\usepackage{xspace}
\usepackage{tikz}
\usepackage{morefloats,subfig,afterpage}
\usepackage{mathrsfs} % script font
\usepackage{verbatim}

%% Using Babel allows other languages to be used and mixed-in easily
%\usepackage[ngerman,english]{babel}
\usepackage[english]{babel}
\selectlanguage{english}


%% Citation system tweaks
\usepackage{cite}
% \let\@OldCite\cite
% \renewcommand{\cite}[1]{\mbox{\!\!\!\@OldCite{#1}}}

%% Maths
% TODO: rework or eliminate maybemath
\usepackage{abmath}
\DeclareRobustCommand{\mymath}[1]{\ensuremath{\maybebmsf{#1}}}
% \DeclareRobustCommand{\parenths}[1]{\mymath{\left({#1}\right)}\xspace}
% \DeclareRobustCommand{\braces}[1]{\mymath{\left\{{#1}\right\}}\xspace}
% \DeclareRobustCommand{\angles}[1]{\mymath{\left\langle{#1}\right\rangle}\xspace}
% \DeclareRobustCommand{\sqbracs}[1]{\mymath{\left[{#1}\right]}\xspace}
% \DeclareRobustCommand{\mods}[1]{\mymath{\left\lvert{#1}\right\rvert}\xspace}
% \DeclareRobustCommand{\modsq}[1]{\mymath{\mods{#1}^2}\xspace}
% \DeclareRobustCommand{\dblmods}[1]{\mymath{\left\lVert{#1}\right\rVert}\xspace}
% \DeclareRobustCommand{\expOf}[1]{\mymath{\exp{\!\parenths{#1}}}\xspace}
% \DeclareRobustCommand{\eexp}[1]{\mymath{e^{#1}}\xspace}
% \DeclareRobustCommand{\plusquad}{\mymath{\oplus}\xspace}
% \DeclareRobustCommand{\logOf}[1]{\mymath{\log\!\parenths{#1}}\xspace}
% \DeclareRobustCommand{\lnOf}[1]{\mymath{\ln\!\parenths{#1}}\xspace}
% \DeclareRobustCommand{\ofOrder}[1]{\mymath{\mathcal{O}\parenths{#1}}\xspace}
% \DeclareRobustCommand{\SOgroup}[1]{\mymath{\mathup{SO}\parenths{#1}}\xspace}
% \DeclareRobustCommand{\SUgroup}[1]{\mymath{\mathup{SU}\parenths{#1}}\xspace}
% \DeclareRobustCommand{\Ugroup}[1]{\mymath{\mathup{U}\parenths{#1}}\xspace}
% \DeclareRobustCommand{\I}[1]{\mymath{\mathrm{i}}\xspace}
% \DeclareRobustCommand{\colvector}[1]{\mymath{\begin{pmatrix}#1\end{pmatrix}}\xspace}
\DeclareRobustCommand{\Rate}{\mymath{\Gamma}\xspace}
\DeclareRobustCommand{\RateOf}[1]{\mymath{\Gamma}\parenths{#1}\xspace}

%% High-energy physics stuff
\usepackage{abhep}
\usepackage{hepnames}
\usepackage{hepunits}
\DeclareRobustCommand{\arXivCode}[1]{arXiv:#1}
\DeclareRobustCommand{\CP}{\ensuremath{\mathcal{CP}}\xspace}
\DeclareRobustCommand{\CPviolation}{\CP-violation\xspace}
\DeclareRobustCommand{\CPv}{\CPviolation}
\DeclareRobustCommand{\LHCb}{LHCb\xspace}
\DeclareRobustCommand{\LHC}{LHC\xspace}
\DeclareRobustCommand{\LEP}{LEP\xspace}
\DeclareRobustCommand{\CERN}{CERN\xspace}
\DeclareRobustCommand{\bphysics}{\Pbottom-physics\xspace}
\DeclareRobustCommand{\bhadron}{\Pbottom-hadron\xspace}
\DeclareRobustCommand{\Bmeson}{\PB-meson\xspace}
\DeclareRobustCommand{\bbaryon}{\Pbottom-baryon\xspace}
\DeclareRobustCommand{\Bdecay}{\PB-decay\xspace}
\DeclareRobustCommand{\bdecay}{\Pbottom-decay\xspace}
\DeclareRobustCommand{\BToKPi}{\HepProcess{ \PB \to \PK \Ppi }\xspace}
\DeclareRobustCommand{\BToPiPi}{\HepProcess{ \PB \to \Ppi \Ppi }\xspace}
\DeclareRobustCommand{\BToKK}{\HepProcess{ \PB \to \PK \PK }\xspace}
\DeclareRobustCommand{\BToRhoPi}{\HepProcess{ \PB \to \Prho \Ppi }\xspace}
\DeclareRobustCommand{\BToRhoRho}{\HepProcess{ \PB \to \Prho \Prho }\xspace}
\DeclareRobustCommand{\X}{\thesismath{X}\xspace}
\DeclareRobustCommand{\Xbar}{\thesismath{\overline{X}}\xspace}
\DeclareRobustCommand{\Xzero}{\HepGenParticle{X}{}{0}\xspace}
\DeclareRobustCommand{\Xzerobar}{\HepGenAntiParticle{X}{}{0}\xspace}
\DeclareRobustCommand{\epluseminus}{\Ppositron\!\Pelectron\xspace}
\DeclareRobustCommand{\protonproton}{\Pproton\APantiproton\xspace}
\renewcommand{\familydefault}{phv}

%% Survey stuff
\usepackage{cite}
\usepackage{afterpage}

\usepackage{array}

\usepackage{colortbl}
%\usepackage[dvipsnames]{xcolor}

\usepackage{hhline}
\usepackage{lscape}

% Tables %
\usepackage{tabularx}
\usepackage{multirow}
\usepackage{hhline}
\renewcommand\tabularxcolumn[1]{m{#1}} %vertical alignment in TabularX
\newcolumntype{C}{>{\centering\arraybackslash}X} % centered X columnType (C)
\newcolumntype{R}{>{\flushright\arraybackslash}X}
\newcommand{\ra}[1]{\renewcommand{\arraystretch}{#1}} % Stretch rows

\usepackage{amssymb}% http://ctan.org/pkg/amssymb
\usepackage{latexsym}
\usepackage{pifont}% http://ctan.org/pkg/pifont
\newcommand{\cmark}{\ding{51}}%
\newcommand{\xmark}{\ding{55}}%
\newcommand{\mapItem} {\color{black!70} \cmark}


\usepackage{mwe}
\usepackage{graphbox} %loads graphicx package
\usepackage{pdflscape}

\usepackage{wrapfig}
\usepackage{multirow, makecell}

\usepackage{stfloats} % bottom table

\usepackage[export]{adjustbox}
\usepackage{subfig}

\makeatletter
\def\BState{\State\hskip-\ALG@thistlm}
\makeatother
%            %

% Psuedocode %
\usepackage{amsmath}
\usepackage{float}
\usepackage{hyperref}
\usepackage{algorithm}
\usepackage[noend]{algpseudocode}
\newcommand{\IfEnd}{\EndIf \State{\textbf{endIf}}}
\newcommand{\ForEnd}[1]{\EndFor \State{\textbf{endFor (#1)}}}
\newcommand{\WhileEnd}{\EndWhile \State{\textbf{endWhile}}}

% plots %
\usepackage{pgfplots}
\usepackage{pgfplotstable}
\usepackage{booktabs}
\usepackage{colortbl} \newcommand{\myrowcolour}{\rowcolor[gray]{0.925}}
\usetikzlibrary{pgfplots.dateplot}
\usepgfplotslibrary{dateplot}
\pgfplotsset{width=7cm,compat=1.13}\usepgfplotslibrary{dateplot}

\pgfplotstableset{% global config, for example in the preamble
  every head row/.style={before row=\toprule,after row=\midrule},
  every last row/.style={after row=\bottomrule},
  fixed,precision=2,
}


%TABLE TEMPLATES
\newcommand*\fillItem{\cellcolor{black!50}}
\newcommand*\fillCellTitleTop{\cellcolor{black!25}}
\newcommand*\fillCellTitle{\cellcolor{black!15}}
\newcommand*\fillCell{\cellcolor{black!50}}
\newcommand*\fillCellSubHeader{\cellcolor{black!05}}
\newcommand*\subVRule{{\color{black!30}\vrule}}
%\newcommand*\altrowColor{\color{MidnightBlue!10}
\def\tabularxcolumn#1{m{#1}}
\newcolumntype{C}{>{\centering\arraybackslash}X}%

\newcommand*{\tabbox}[2][t]{%
    \vspace{0pt}\parbox[#1][3.7\baselineskip]{1cm}{\strut#2\strut}}
    
    \newcolumntype{R}[2]{%
    >{\adjustbox{angle=#1,lap=\width-(#2)}\bgroup}%
    l%
    <{\egroup}%
}
\newcommand*\rot{\multicolumn{1}{R{90}{1em}}}% no optional argument here, please!

\usepackage{enumitem}

%\usepackage{subfigure}

\newcolumntype{C}{>{\centering\arraybackslash}X}
\usepackage{stackengine}
\usepackage{multirow}

\usepackage{listings}
\definecolor{codegreen}{rgb}{0,0.6,0}
\definecolor{codegray}{rgb}{0.5,0.5,0.5}
\definecolor{codepurple}{rgb}{0.6,0.3,0.12}
\definecolor{codepurp}{rgb}{0.13,0.0,0.92}
\definecolor{backcolour}{rgb}{0.95,0.95,0.92}

\lstdefinestyle{mystyle}{
    backgroundcolor=\color{backcolour},   
    commentstyle=\color{codegreen},
    keywordstyle=\color{codepurp},
    numberstyle=\tiny\color{codegray},
    stringstyle=\color{codepurple},
    basicstyle=\footnotesize,
    breakatwhitespace=false,         
    breaklines=true,                 
    captionpos=b,                    
    keepspaces=true,                 
    numbers=left,                    
    numbersep=5pt,                  
    showspaces=false,                
    showstringspaces=false,
    showtabs=false,                  
    tabsize=2,
    morekeywords={Point,LineSegment,Circle,Triangle,AABB}
}
 
\lstset{style=mystyle}

\newcommand{\cons}{{ \color{red} \ding{117}}}
\usepackage{pdfpages}
\newcommand{\header}{\cellcolor{black!15}}
\newcommand{\specialRule}{\hhline{>{\arrayrulecolor{black!15}}->{\arrayrulecolor{black}}------}}
\newcommand{\highlight}[1]{{\color{red}#1}}
\newcommand{\greenfont}{\color{black!20!olive}}
\newcommand{\arrow}{\ding{213}}